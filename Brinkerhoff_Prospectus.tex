% !TEX TS-program = lualatex
% !TEX encoding = UTF-8 Unicode

\documentclass[12pt, letterpaper]{article}

%%BIBLIOGRAPHY- This uses biber/biblatex to generate bibliographies according to the 
%%Unified Style Sheet for Linguistics
\usepackage[main=american, german]{babel}% Recommended
\usepackage{csquotes}% Recommended
\usepackage[backend=biber,
            style=unified,
            maxcitenames=3,
            maxbibnames=99,
            natbib,
            url=false]{biblatex}
\addbibresource{Prospectus.bib}
\setcounter{biburlnumpenalty}{100}  % allow URL breaks at numbers
%\setcounter{biburlucpenalty}{100}   % allow URL breaks at uppercase letters
%\setcounter{biburllcpenalty}{100}   % allow URL breaks at lowercase letters

%%TYPOLOGY
\usepackage[svgnames]{xcolor} % Specify colors by their 'svgnames', for a full list of all colors available see here: http://www.latextemplates.com/svgnames-colors
%\usepackage[compact]{titlesec}
%\titleformat{\section}[runin]{\normalfont\bfseries}{\thesection.}{.5em}{}[.]
%\titleformat{\subsection}[runin]{\normalfont\scshape}{\thesubsection}{.5em}{}[.]
\usepackage[hmargin=1in,vmargin=1in]{geometry}  %Margins
\usepackage{graphicx} % 
\usepackage{stackengine} %Package to allow text above or below other text, Also helpful for HG weights 
\usepackage{fontspec} %Selection of fonts must be ran in XeLaTeX
\usepackage{amssymb} %Math symbols
\usepackage{amsmath} % Mathematical enhancements for LaTeX
\usepackage{setspace} %Linespacing
\usepackage{multicol} %Multicolumn text
\usepackage{enumitem} %Allows for continuous numbering of lists over examples, etc.
\usepackage{multirow} %Useful for combining cells in tablesbrew 
\usepackage{booktabs}
\usepackage{hanging}
\usepackage{fancyhdr} %Allows for the 
\pagestyle{fancy}
\fancyhead[L]{\textit{Phonetic Realization and Phonological Representation}} 
\fancyhead[R]{\textit{\today}} 
\fancyfoot[L,R]{} 
\fancyfoot[C]{\thepage} 
\renewcommand{\headrulewidth}{0.4pt}
\setlength{\headheight}{14.5pt} % ...at least 14.49998pt
% \usepackage{fourier} % This allows for the use of certain wingdings like bombs, frowns, etc.
% \usepackage{fourier-orns} %More useful symbols like bombs and jolly-roger, mostly for OT
\usepackage[colorlinks,allcolors={black},urlcolor={blue}]{hyperref} %allows for hyperlinks and pdf bookmarks
% \usepackage{url} %allows for urls
% \def\UrlBreaks{\do\/\do-} %allows for urls to be broken up
\usepackage[normalem]{ulem} %strike out text. Handy for syntax
\usepackage{tcolorbox}
\usepackage{datetime2}

%%FONTS
\setmainfont{Libertinus Serif}
\setsansfont{Libertinus Sans}
\setmonofont[Scale=MatchLowercase]{Libertinus Mono}

%%PACKAGES FOR LINGUISTICS
%\usepackage{OTtablx} %Generating tableaux with using TIPA
% \usepackage[noipa]{OTtablx} % Use this one generating tableaux without using TIPA
%\usepackage[notipa]{ot-tableau} % Another tableau drawing packing use for posters.
% \usepackage{linguex} % Linguistic examples
% \usepackage{langsci-linguex} % Linguistic examples
% \usepackage{langsci-gb4e} % Language Science Press' modification of gb4e
% \usepackage{langsci-avm} % Language Science Press' AVM package
% \usepackage{tikz} % Drawing Hasse diagrams
% \usepackage{pst-asr} % Drawing autosegmental features
% \usepackage{pstricks} % required for pst-asr, OTtablx, pst-jtree.
% \usepackage{pst-jtree} %Syntax tree draawing software
% \usepackage{tikz-qtree} % Another syntax tree drawing software. Uses bracket notation.
% \usepackage[linguistics]{forest} % Another syntax tree drawing software. Uses bracket notation.
% \usepackage{ling-macros} % Various linguistic macros. Does not work with linguex.
% \usepackage{covington} % Another linguistic examples package.
\usepackage{leipzig} % Offers support for Leipzig Glossing Rules

%%LEIPZIG GLOSSING FOR ZAPOTEC
\newleipzig{el}{el}{elder} % Elder pronouns
\newleipzig{hu}{hu}{human} % Human pronouns
\newleipzig{an}{an}{animate} % Animate pronouns
\newleipzig{in}{in}{inanimate} % Inanimate pronouns
\newleipzig{pot}{pot}{potential} % Potential Aspect
\newleipzig{cont}{cont}{continuative} % Continuative Aspect
% \newleipzig{pot}{pot}{potential} % Potential Aspect
\newleipzig{stat}{stat}{stative} % Potential Aspect
\newleipzig{and}{and}{andative} % Andative Aspect
\newleipzig{ven}{ven}{venative} % Venative Aspect
% \newleipzig{res}{res}{restitutive} % Restitutive Aspect
\newleipzig{rep}{rep}{repetitive} % Repetitive Aspect

%%TITLE INFORMATION
\title{Brinkerhoff Dissertation Prospectus}
\author{Mykel Loren Brinkerhoff}
\date{\today}

%%MACROS
\newcommand{\sub}[1]{\textsubscript{#1}}
\newcommand{\supr}[1]{\textsuperscript{#1}}
\providecommand{\lsptoprule}{\midrule\toprule}
\providecommand{\lspbottomrule}{\bottomrule\midrule}
\newcommand{\fittable}[1]{\resizebox{\textwidth}{!}{#1}}

\makeatletter
\renewcommand{\paragraph}{%
    \@startsection{paragraph}{4}%
    {\z@}{0ex \@plus 1ex \@minus .2ex}{-1em}%
    {\normalfont\normalsize\bfseries}%
}
\makeatother
\parindent=10pt


\begin{document}

%%If using linguex, need the following commands to get correct LSA style spacing
%% these have to be after  \begin{document}
    % \setlength{\Extopsep}{6pt}
    % \setlength{\Exlabelsep}{9pt}%effect of 0.4in indent from left text edge
%%

%% Line spacing setting. Comment out the line spacing you do not need. Comment out all if you want single spacing.
% \doublespacing
\onehalfspacing

\begin{center}
    {\Large \textbf{Phonetic Realization and Phonological Representation of Voice Quality in Santiago Laxopa Zapotec}}
    \vspace{6pt}

    Mykel Loren Brinkerhoff
\end{center}
%\maketitle
%\maketitleinst
\thispagestyle{fancy}

% \tableofcontents

%------------------------------------
\section{Proposal} \label{sec:Introduction}
%------------------------------------

My dissertation addresses how phonology and phonetics account for differences in voice quality. 
Voice quality describes how the larynx can be manipulated during speech to affect how the sounds are produced. Most commonly, in speech, these manipulations have a breathy or creaky voice quality \citep{eslingVoiceQualityLaryngeal2019}. 
In some languages, like English, the amount of breathiness or creakiness produced when we speak does not significantly affect what we say. 
However, this is different in other languages across the world. For example, in many Mesoamerica and Southeast Asian indigenous languages, voice quality differences can drastically change a word's meaning \citep{espositoCrossLinguisticPatterns2020}.
In the endangered Zapotec language Santiago Laxopa Zapotec (SLZ) spoken by approximately 1000 speakers in the municipality of Santiago Laxopa, Ixtlán, Oaxaca, Mexico \citep{adlerAcousticsPhonationTypes2016, adlerDerivationVerbInitiality2018, brinkerhoffDownstepSantiagoLaxopaMFM, foleyForbiddenCliticClusters2018, SantiagoLaxopaEconomy, sichelFeaturalLifeNominals2020, silva-roblesElicitingAssociatedMotion2022} there are four unique voice qualities breathy, checked, rearticulated, and modal. 
Breathy vowels are realized with a breathy voice during any point of the vowel. 
Checked and rearticulated vowels are realized with creaky voice but in different locations within the vowel (i.e., checked vowels have creakiness at the end while rearticulated vowels have creakiness at the middle).
Modal vowels, or the default vowel, do not have any laryngeal modification. 

My dissertation will address several questions about how these voice quality distinctions are produced and what they can tell us about how these sounds are organized in the sound system of SLZ. The first concerns the phonetic realization or how SLZ makes these sounds. Initial investigations into these voice qualities by \citet{adlerAcousticsPhonationTypes2016} and \citet{brinkerhoffResidualH1MeasureInPreparation} have revealed some information about this question, mainly what acoustic measures best capture the contrasts between these qualities. \citet{adlerAcousticsPhonationTypes2016} showed that spectral tilt can account for the four voice quality distinctions in SLZ. This acoustic measure looks at the speech signal's relative amplitude of sound waves \citep{fischer-jorgensenPhoneticAnalysisBreathy1968}. \citeauthor{adlerAcousticsPhonationTypes2016} found that two different spectral tilt measures can capture the distinctions. However, recent work in \citet{brinkerhoffResidualH1MeasureInPreparation} on ten SLZ speakers shows that most spectral tilt measures are not good at capturing the voice quality distinctions. Instead, we found that most spectral tilt measures could only catch some differences. Instead, the acoustic measure called Strength of Excitation, which measures how strongly the vocal folds are vibrating \citep{murtyEpochExtractionSpeech2008, mittalStudyEffectsVocal2014}, and a new acoustic measure proposed by \citet{chaiH1H2Acoustic2022}, Residual H1, better capture the voice quality contrasts in SLZ. This aligns with more recent work by \citet{chaiH1H2Acoustic2022} and \citet{zhangCauseeffectRelationshipVocal2016, zhangMechanicsHumanVoice2016}, where they show and discuss how spectral tilt measures are not as robust as previously thought. My dissertation expands on this discussion about why spectral tilt measures fail to capture the contrasts in SLZ and the role that Strength of Excitation, residual H1, and other acoustic measures play in establishing and describing these voice quality distinctions. This phonetic knowledge is critical in explaining how these voice qualities are produced.

It is commonly accepted that the sounds used by language are organized and stored in an abstract language system we call the phonological component of mental grammar. Different linguistic theories argue that there is little to no phonetic input \citep{reissSubstanceFreePhonology2017}, whereas others say that phonetics and phonology are entirely unified with total overlap \citep{flemmingScalarCategoricalPhenomena2001}. One of the questions that my dissertation will address is to what extent the data from SLZ voice quality can shed light on this interaction, particularly how it relates to the representation of these sounds in the phonological component of mental grammar. These phonological representations form the building blocks from which our understanding of how sounds pattern and are organized into cohesive systems. There is debate about how information is stored in these phonological representations. Some linguists argue that only a single form is stored (e.g., \cite{albrightIdentificationBasesMorphological2002}). Others argue that instead, we have a set of multiple forms stored (e.g., \cite{mascaroExternalAllomorphyLexical2007}) or all the actual forms spoken or heard (e.g., exemplar theory, \cite{ernestusIntraparadigmaticEffectsPerception2007}). Essentially, these views ask how abstract these phonological representations could be for the grammar to operate. My dissertation will contribute to the ongoing debate about the nature of these phonological representations. 


% Zapotec is an ideal candidate for addressing these questions because of its many varieties of complex voice quality systems \cite{silvermanLaryngealComplexityOtomanguean1997, ariza-garciaPhonationTypesTones2018}. This is especially true when multiple phonological distinctions rely on the same phonetic realization (i.e., creaky voice in checked and rearticulated vowels). Arellanes Arellanes (2009) presents one such theory to explain San Pablo Güilá Zapotec’s checked and rearticulated vowels with the claim that these vowels are associated with strong or weak glottal stops, respectively. 

% %------------------------------------
% \section{Outline and structure of dissertation} \label{sec:Outline}
% %------------------------------------

% I anticipate that the dissertation will contain the following information in the following chapters.

% \begin{enumerate}
%     \item Introduction
%     \begin{itemize}
%         \item Overview of what tone, voice quality, and their interactions are
%         \item Introduce research questions and their importance
%         \item Provide roadmap for dissertation
%     \end{itemize}
%     \item Language Background
%     \begin{itemize}
%         \item Introduce the main language of study
%         \item A brief phonetic and phonological description of SLZ
%         \item Description of the tones
%         \item Description of the voice quality
%     \end{itemize}
%     \item Theoretical Background
%     \begin{itemize}
%         \item Discuss phonetics-phonology interface
%         \item Introduce and discuss Laryngeal Complexity Hypothesis
%         \item Talk about current ways of accounting for tone and voice quality 
%         \begin{itemize}
%             \item Autosegmental–metrical theory 
%             \item Phonological features
%         \end{itemize}
%         \item Current accounts for dealing with timing issues in phonology
%         \begin{itemize}
%             \item Gestural Phonology
%             \item Q-theory
%         \end{itemize}
%     \end{itemize}
%     \item Acoustic Study of Voice Quality in SLZ
%     \begin{itemize}
%         \item Production study of Voice Quality based on data Maya and I collected. 
%         \begin{itemize}
%             \item This is essentially parts of the QE that I have been expanding on over the summer. 
%         \end{itemize}
%         \item Perception study of Voice Quality
%         \begin{itemize}
%             \item This will give a more complete picture about the acoustics of voice quality and will help show what people are relying on when listening to these voice quality contrasts. 
%         \end{itemize}
%     \end{itemize}
%     \item Measuring Laryngeal Complexity
%     \begin{itemize}
%         \item Generalized Additive Mixed Model analysis on the phonation types. 
% 		\begin{itemize}
%             \item What are GAMMs
%             \item Why use GAMMs?
%             \item Do this on f0 and the measures that seem reliable
%         \end{itemize}
%         \item Phasing is probably only important for checked and laryngealized vowels
%         \item What does this do for me?
%     \end{itemize}
%     \item The Phonological Structures of Voice Quality and Tone
%     \begin{itemize}
%         \item Develop a framework for explaining the timing differences that we observe
%         \item Develop a framework that accounts for the interaction of tone and voice quality
%     \end{itemize}
%     \item Implications
%     \begin{itemize}
%         \item Discuss the implications for my proposal to other languages
%         \item How does it account for voice quality and tone interactions in other systems
%         \begin{itemize}
%             \item Things like tone and voice quality being dependent on each other. 
%         \end{itemize}
%     \end{itemize}
%     \item Conclusion
%     \begin{itemize}
%         \item Summary of the dissertation, key takeaway, etc.
%     \end{itemize}
% \end{enumerate}

%------------------------------------
\section{Previous and current work} \label{sec:Work}
%------------------------------------

The dissertation grew from work I conducted for my second qualifying paper, a preliminary description and analysis of voice quality based on consultant data collected during the COVID-19 lockdowns. This was based on two speakers that live in the Monterey Bay area. This preliminary data showed that what was reported in \citep{adlerAcousticsPhonationTypes2016} might be more complicated than previously assumed. In the summer of 2022, I traveled to Oaxaca. With the help of a colleague, I collected fresh data from 15 individuals doing word list elicitations with lexical items that had each of the four different voice qualities present, as well as new vocabulary items, stories, and recipes. Over the past two years, I have been annotating the data and running small-scale acoustic analyses on a portion of the data. The result of this was my qualifying exam. During the summer of 2024, the annotation was completed on the word lists, and a new analysis was conducted, which showed, as mentioned above, that traditional spectral tilt measures were not as robust on this large-scale analysis. This research was submitted to a conference and is being prepared for publication \citep{brinkerhoffResidualH1MeasureInPreparation}. Data collection is completed; however, this does not rule out the possibility that I might need to return to Laxopa to collect additional data. I am researching the theoretical implications of the results, as discussed above. I am currently reading and evaluating the literature on phonological representations. A timeline for the remaining steps is provided below.

%------------------------------------
\section{Timeline for dissertation} \label{sec:Timeline}
%------------------------------------
\begin{table}[!h]
    \centering
    \begin{tabular}{ll}
    \lsptoprule   
    Milestone & Date \\ 
    \hline
    Finish theoretical research & May 2024 \\
    Start first draft of dissertation & May 2024 \\
    Conference presentation at LabPhon & June 2024 \\
    Potential research travel to Oaxaca & July 2024 \\
    Process additional data if needed & August--October 2024 \\
    First draft completed & December 2024  \\
    Revisions to dissertation & December 2024--January 2025 \\
    Present at LSA and SSILA & January 2025 \\
    Completed defense draft & February 2025 \\
    Defend Dissertation & March 2025 \\
    Revisions completed & May 2025 \\
    Submit Dissertation & May 2025 \\ 
    \lspbottomrule
    \end{tabular}
\end{table}

%------------------------------------
%BIBLIOGRAPHY
%------------------------------------

%\singlespacing
%\nocite{*}
\printbibliography[heading=bibintoc]

\end{document} 